\section{Conclusion}
This work has set out to find out which models are the best suited for long term modeling of solar PV, solar TH and wind hourly capacity factor. This aims to further the literature for long term modeling -- given the usual focus on short or medium term for most papers -- in order to better aid electrical grid planners and operators to integrate more solar and wind energy resources. In order to do this, 12 different models belonging to 4 different model families are implemented and evaluated along 9 different metrics. These metrics aim to evaluate different aspects of the modeled data: overall distribution fit, temporal consistency, multivariate coherence and extreme value fit. 

After a careful analysis it has become apparent how different models are better suited for different variables and different characteristics of such variables. For the solar PV capacity factor the N-HiTS emerges as the best overall model due to its ability to best capture and leverage frequency components, which are the main driving factor of this series. % The iTransformer -- a transformer based model --, the SARIMAX -- a statistical model -- and the TimeMixer -- another neural network based model -- tag closely behind in terms of performance.
As for the solar thermal capacity factor, due to its more complex self and cross dependence it is best modeled by the iTransformer, which thanks to its architecture is better suited for capturing more complex and long run patterns. 
The wind series is the most different to the other two. Due to its low signal-to-noise ratio this series is best modeled by a simple statistical model, the VARMAX. This simple statistical model however is able to leverage multivariate information for modeling wind data. The NN based models N-BEATS and TimeMixer -- very good performers for the previous two series as well -- are however the ones that best capture cross-variable relationships for this series. 

The work also demonstrates the usefulness of implementing the SERA as a custom loss function for extreme value modeling in the wind series. This loss function properly steers a sample XGBoost model to better predict extreme wind events in the upper tail of the distribution at the expense of worse overall distribution fit. As an unintended side effect, the temporal consistency metrics also improved for the upper tail biased model. These results are however not replicated for the lower tail of the distribution, where the base model remains the best predictor for extreme values. 

This work also highlights the shortcomings of the implemented models in simulating wind data. The performance of the best model is comparatively much worse than that of the best model of the solar PV and solar TH series. After close analysis, it is hypothesized that the gap in performance could be improved by choosing a loss function that does not penalize incorrect point wise predictions so harshly.