\section{Literature Review}
% A review and taxonomy of wind and solar energy forecasting methods based on deep learning
% State-Of-The-Art Solar Energy Forecasting Approaches: Critical Potentials and Challenges
% Energy Forecasting: A Comprehensive Review of Techniques and Technologies

Given the great and increasing importance of renewable energy, it is of no surprise that many scholars have already shown great interest in all aspects surrounding it. From the development of the technology, to the analysis of its use, there is an incredible amount of papers regarding the topic. Given this astonishing breadth, this literature review will focus strictly on modeling and forecasting papers.

Regarding the data used by the model, two main types of models have been observed. There are models which predict power data based on physical data such as wind speed, temperature, rainfall, humidity, etc. for solar power such as in \cite{wang_urquhart_kleissl_2019} with other models even using images obtained from satellites like in \cite{shen_yao_wen_he_jiang_2018}. On the other hand, there are models which try to predict the target variable's values from historical data or non-physical exogenous variables. These will be the main focus of the literature review given how this is the data available for the present work. Finally, there are also some hybrid models leveraging both.  

There is also great diversity with respect to the predicted variable. With respect to wind, some works focus on predicting wind speed as in \cite{he_nie_guo_wang_2020} while others on wind power as in \cite{wang_tao_hu_zeng_2020}. For the solar resource, predicted variables include solar irradiance as in \cite{alzahrani_shamsi_cihan_ferdowsi_2017} also expressed as solar irradiation as in \cite{learning_2017}, solar power as in \cite{suresh_janik_guerrero_leonowicz_sikorski_2020} or more specifically solar PV power as in \cite{sharadga_hajimirza_balog_2020} or solar thermal power as in \cite{wang_guo_zhang_song_duan_duan_2020}.

% A good initial view of papers regarding these models is given by \cite{ghadah_alkhayat_mehmood_2021}. This review surveys both solar and wind energy forecasting. 
% \subsection{Data preprocessing}
Starting with the data preprocessing methods, most papers use Min-Max scaling like in \cite{ju_sun_chen_zhang_zhu_rehman_2019} where it is used in the forecasting of solar PV to normalize all strictly positive data, while Zero-Mean normalization is used on data with positive and negative values. Regarding wrong or missing value handling most papers report removing such records, although some papers such as \cite{peng_peng_fu_lu_tang_wang_li_2020} use linear imputation. Regarding the removal of outliers, some papers such as \cite{sharadga_hajimirza_balog_2020} use the Hampel filter while others like \cite{zang_cheng_ding_cheung_wang_wei_sun_2020} interpret as outliers any values exceeding three times the standard deviation and impute it with a quadratic spline.  
Some papers also use data augmentation for the model to be able to focus on specific weather conditions, like \cite{wang_zhang_liu_yu_pang_nevenshafie2018} where 10 categories of weather are uncovered and a Generative Adversarial Network (GAN) is used to generate more data for some of these categories to solve the data imbalance present in the dataset. In some other cases some additional features are calculated and included in the dataset, like in \cite{yin_ou_huang_meng_2019} where the sine and cosine of the wind direction is calculated and included.
Other techniques found in the literature are the clustering of the data to divide it into seasons or weather conditions like in \cite{alzahrani_shamsi_cihan_ferdowsi_2017} or to assign labels such as in \cite{wang_xuan_zhen_li_wang_shi_2020}.

Regarding the prediction time horizon there is also great diversity in the published literature. Some papers like \cite {hu_zhang_zhou_2015} or \cite{chen_zhu_li_zhu_shi_li_duan_liu_2019} focus on short term forecast horizons on the order of 10 minutes to a few hours.  Others like \cite{aly_2020} focus on a medium horizon of 1 week for its wind speed and power prediction. There are few where the horizon is considered long term like \cite{ray_shah_islam_islam_2020}, which predicts PV power with a multi-year horizon. However, this paper like others found with a long term horizon use a monthly resolution, so the forecast horizon, although long in the temporal sense, is not that many steps ahead. 

As for the type of model used, it is quite astonishing the diversity of models that have been applied to the task of forecasting solar and wind power or related variables. Within the statistical methods' category, some authors like \cite{pedro_coimbra_2012} implement an ARIMA model, \cite{m_bouzerdoum_mellit_pavan_2013} extends it to a SARIMA model combined with an SVM or \cite{haddad_nicod_2019} implements a vanilla SARIMA model, \cite{jain_none_behera_2022} implements a VARMAX used for wind power forecasting by forecasting it together with other variables determined to be causal through a Granger test. As additional less conventional statistical methods, \cite{delgado_estefan_neill_carrillo_andrade_2024} leverages a Hidden Markov Model (HMM) combined with an LSTM for PV power generation forecasting, \cite{frimane_munkhammar_van_meer_2022} uses an infinite HMM for short term solar irradiance forecasting and \cite{wang_wu_cao_hong_2022} uses a Markov switching GARCH-MIDAS model for forecasting renewable stock volatility in short and long term horizons. 
Regarding Machine Learning based models, \cite{abuella_chowdhury_2016} uses Support Vector Regression (SVR) for a one-year ahead hourly granularity forecast of solar power, others like \cite{aksoy_istemihan_genc_2023} implements three different gradient boosting models for solar power plant generation forecasting. 
However, the most explored type of models is without a doubt the Deep Learning or Neural Network based models. The models that have been implemented for solar and wind generation start at plain Feed Forward Neural Networks (FFNN) as in \cite{pedro_coimbra_2012}. Convolutional Neural Networks which are usually used for 2 or 3 dimensional data can be adapted for 1 dimension and used for solar PV power forecasting as in \cite{huang_kuo_2019}. Recurrent Neural Networks (RNN) are specially well suited for these types of forecasting tasks given their ability to retain temporal relationships. Within the family of RNN, Long Short Term Memory networks (LSTM) are implemented by itself as in \cite{abdel_nasser_mahmoud_2017} or \cite{hossain_mahmood_2020}, they are stacked as in \cite{liang_nguyen_jin_2018} or implemented in an autoencoder architecture as in \cite{suresh_janik_guerrero_leonowicz_sikorski_2020}. Other RNN architectures like Gated Recurrent Unit (GRU) networks are also implemented in \cite{learning_2017} for solar irradiation forecasting. Deep Belief Networks (DBN) in which hidden layers have connection between layers but not between units in a given layer are also used for forecasting wind speed in \cite{lin_pai_ting_2019} or for solar PV power in \cite{neo_teo_woo_t_logenthiran_sharma_2017}. 
More recently, given the huge popularity of Transformer models for speech processing, some papers exploring their use for renewable energy forecasting have appeared, like \cite{kim_obregon_park_jung_2024} which uses a transformer architecture for solar PV power forecasting.
A special approach used by some regarding deep learning is that of transfer learning, where a pretrained model is employed on a different target domain like \cite{mylona_doukas_2022} using TL for solar power forecasting.
Another big family of models that is widely explored in the literature is that of ensemble models. These models combine two or more basic models through some weighting scheme, stacking architecture, voting model or more sophisticated combination method. An example of a paper using such models is the one implemented by \cite{liu_mi_li_duan_xu_2019} which combines a CNN, GRU network and SVR for wind speed forecasting.

Regarding evaluation metrics, there is also some diversity in the kind of metrics that are used. The work by \cite{aristeidis_tjortjis_2024} gives a good overview on some widely used metrics, mentioning the coefficient of determination ($R^2$), Mean Absolute Error (MAE), Mean Absolute Percentage Error (MAPE), Root Mean Squared Error (RMSE), Coefficient of Variation of the Root Mean Squared Error (CVRMSE) and Normalized Root Mean Squared Error (NRMSE).
% \subsection{Overview of renewable energy in Spain}
% \subsection{Solar power generation}
% \subsection{Wind power generation}
% \subsection{Modeling techniques}
% \subsection{Performance metrics for model evaluation}