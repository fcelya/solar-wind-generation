\section{Results}
After having described all the different implemented models, having formulated a hypothesis about which should be able to outperformed and explained how the models are trained and evaluated, it is now time to present and analyze the obtained results. This section will have four subsection. The first three will focus on each of the three capacity factors, analyzing separately which models outperform and why in each of them. The final section will be devoted to analyzing whether the custom loss function successfully outperforms the base loss function in extreme value modeling. 
\subsection{Solar PV}
The first results that will be presented are those of the photovoltaic solar capacity factor. The results will be presented for all metrics for all evaluated models in a similar way as was presented in \autoref{table:eval-metrics-test-validation-solar-th}: \nameref{table:eval-metrics-test-validation-solar-th}.

The metrics of the validation set obtained in the \nameref{Model training and validation} section will also be included as a reference point. 

\begin{table}[ht]
    \footnotesize
    \begin{tabular}[l]{r|c|ccc|c}
        \toprule
        \textbf{Solar PV} &Benchmark&Regression&SARIMAX&VARMAX&SVM  \\ 
        \midrule            
        Cramer von Mises&29.34&364.6&93.27&29.21&223.3 \\ 
        KL divergence&0.0002762&0.4844&0.1905&0.0002732&0.1662 \\ 
        ACF$_\xi$ distance&0.0127&0.1201&0.1089&0.01338&0.2289 \\ 
        \midrule
        CCMD&0.01117&0.3291&0.1044&0.01117&0.8375 \\ 
        CCF$_\xi^{Solar TH}$ distance&0.009855&0.1232&0.08045&0.009355&0.1484 \\ 
        CCF$_\xi^{Wind}$ distance&0.01187&0.07759&0.04984&0.01203&0.1091 \\ 
        \midrule
        CVaR$^+$ distance&-0.02966&8.394&0.08376&0.03057&0.6503 \\ 
        Tail dependence coefficient$^+$&0.3676&0.2671&0.3174&0.3676&0.1187 \\ 
        Return level distance$^-$&0.01537&1.857e+07&-0.3411&-0.0004157&0.7803 \\ 
        \bottomrule
    \end{tabular}
\end{table}
\begin{table}[ht]
    \footnotesize
    \begin{flushright}
    \begin{tabular}[r]{c|ccc|cc}
        \toprule
        XGBoost&N-BEATS&N-HiTS&TimeMixer&Informer&iTransformer \\ 
        \midrule            
        747.4&69.43&47.81&52.58&485.8&131.1 \\
        1.586&0.3452&0.003905&0.04968&1.302&0.01814 \\
        0.1802&0.1018&0.06985&0.04528&0.2429&0.05605 \\
        \midrule
        0.3549&0.1959&0.4296&0.05632&0.5675&0.2093 \\
        0.3242&0.0481&0.1829&0.0826&0.1649&0.0688 \\
        0.8558&0.02316&0.01634&0.07532&0.06594&0.08443 \\
        \midrule
        -0.7125&1.037&0.003395&0.1888&-0.6828&0.1683 \\
        0.03397&0.2831&0.3265&0.2032&0.04338&0.3699 \\
        -0.9312&4.802&-0.1408&0.7089&-0.02178&0.2181 \\
        \bottomrule
    \end{tabular}
    \end{flushright}
    \caption{Results of the different models for Solar PV\label{long}}
    \label{table:results-solar-pv}
\end{table}

Given the great number of metrics and the difficulty of easily interpreting them, here is another table with the rank of the model for each metric. That is, for each metric, each model will be shown as the 1st, 2nd, 3rd... best model. The benchmark has also been included in this ranking. In bold, the best model -- apart from the benchmark -- is highlighted. 

\begin{table}[ht]
    \footnotesize
    \begin{tabular}[l]{r|c|ccc|c}
        \toprule
        \textbf{Solar PV} &Benchmark&Regression&SARIMAX&VARMAX&SVM \\ 
        \midrule            
        Cramer von Mises&(2)&(9)&(6)&(1)&(8) \\
        KL divergence&(2)&(9)&(7)&(1)&(6) \\
        ACF$_\xi$ distance&(1)&(8)&(7)&(2)&(10) \\
        \midrule
        CCMD&(2)&(7)&(4)&(1)&(11) \\
        CCF$_\xi^{Solar TH}$ distance&(2)&(7)&(5)&(1)&(8) \\
        CCF$_\xi^{Wind}$ distance&(1)&(8)&(5)&(2)&(10) \\
        \midrule
        CVaR$^+$ distance&(2)&(11)&(4)&(3)&(7) \\
        Tail dependence coefficient$^+$&(10)&(5)&(7)&(10)&(3) \\
        Return level distance$^-$&(2)&(11)&(6)&(1)&(8) \\
        \bottomrule
        Total&(42)&(139)&(97)&(44)&(151) \\
        \bottomrule
    \end{tabular}
\end{table}
\begin{table}[ht]
    \footnotesize
    \begin{flushright}
    \begin{tabular}[r]{c|ccc|cc}
        \toprule
        XGBoost&N-BEATS&N-HiTS&TimeMixer&Informer&iTransformer \\ 
        \midrule            
        (11)&(5)&(3)&(4)&(10)&(7) \\
        (11)&(8)&(3)&(5)&(10)&(4) \\
        (9)&(6)&(5)&(3)&(11)&(4) \\
        \midrule
        (8)&(5)&(9)&(3)&(10)&(6) \\
        (11)&(3)&(10)&(6)&(9)&(4) \\
        (11)&(4)&(3)&(7)&(6)&(9) \\
        \midrule
        (9)&(10)&(1)&(6)&(8)&(5) \\
        (1)&(6)&(8)&(4)&(2)&(11) \\
        (9)&(10)&(4)&(7)&(3)&(5) \\
        \bottomrule
        (149)&(94)&(88)&(88)&(137)&(93) \\
        \bottomrule
    \end{tabular}
    \end{flushright}
    \caption{Results of the rank of the different models for Solar PV\label{long}}
    \label{table:results-rank-solar-pv}
\end{table}

At the bottom of \autoref{table:results-rank-solar-pv} a Total can be found. This total is the sum of the rank of the model accross all metrics and it can be seen as a very simple proxy of the overall goodness of the model across all domains -- distribution fit, multivariate coherence and extreme value fit. 

\subsection{Solar TH}
\subsection{Wind}
\subsection{Custom loss function}